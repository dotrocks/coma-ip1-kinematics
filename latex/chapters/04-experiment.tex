\chapter{Experimental Part}

\section{Projectile Shot In An Area With Air Resistance}
With air resistance in mind, we can’t easily make calculations of the movement. For this purpose, we have made 4 different projectile shots and tried to make calculations according to their movement.

\begin{table}[h]
	\centering
	\begin{tabular}{|c|c|c|c|c|}
		\hline
		      & $15^{\circ}$       & $30^{\circ}$       & $45^{\circ}$       & $50^{\circ}$         \\ \hline
		L     & 100.21 cm          & 97.41 cm           & 58.25 cm           & 58.13 cm             \\ \hline
		H     & 3.56 cm            & 7.96 cm            & 9.43 cm            & 11.75 cm             \\ \hline
		$v_0$ & $367 \frac{cm}{s}$ & $312 \frac{cm}{s}$ & $322 \frac{cm}{s}$ & $366.2 \frac{cm}{s}$ \\ \hline
	\end{tabular}
	\caption{Projectile Shot With Air Resistance}
\end{table}

\begin{itemize}
	\item Mass of the projectile: $2.7g$
	\item Average acceleration: $424 \frac{cm}{s^2}$
\end{itemize}

\noindent Due to air resistance, rolling motion of the projectile and other factors, real results are different from the theoretical results. Both $L$ and $H$ is much smaller than is should be because of these factors.

You can access experiment video via this url:

\url{https://youtu.be/2aXTh4kZMRA}
