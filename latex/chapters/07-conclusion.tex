\chapter{Conclusion}

In conclusion, this experiment aimed to investigate projectile motion in real-life
scenarios, taking into account the effects of air resistance. The theoretical tasks
involved deriving mathematical equations to describe the initial angle ($\theta$) and
the envelope curve ($y = f(x)$) of projectile motion. Calculus I task focused on
determining the maximum range ($R$) and height ($H$) for all possible initial angles
$\theta$, along with analyzing the envelope curve function and its derivatives.

The experimental tasks included measuring the average initial velocity ($v_0$) of
the spring gun, obtaining real measurements of distance ($L$) and height ($H$) for
different initial angles $\theta$, and evaluating the mass ($m$) of the ball to determine the
average acceleration component due to air resistance.

Air resistance has a significant impact on the motion of projectiles. By
understanding the effects of air resistance, we can more accurately predict the
trajectory of objects moving through the air.

MATH103 tasks involved matrix operations, including finding the inverse of
matrix A using the Gauss-Jordan method, determining elimination matrices, and
performing LU factorization. Additionally, a linear system was created to model
the experimental data.

In the EE101 tasks, a C++ program was developed to facilitate matrix
calculations, ensuring user-friendly input for matrix size and elements, and
providing the inverse matrix using the Gauss-Jordan method.

Through these combined efforts, the experiment provided a comprehensive
understanding of projectile motion, integrating theoretical concepts with
practical measurements. The results obtained from the mathematical models and
experimental data were compared, shedding light on the accuracy of theoretical
predictions in real-world scenarios. The designed C++ program demonstrated
the practical applicability of computational tools in solving complex
mathematical problems related to physics and engineering. Overall, this
experiment not only enhanced our understanding of projectile motion but also
underscored the importance of considering factors such as air resistance in real
world physics applications.
